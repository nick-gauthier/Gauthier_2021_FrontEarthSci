\documentclass[fleqn,10pt]{wlscirep}
\usepackage[utf8]{inputenc}
\usepackage[T1]{fontenc}
\title{Drought variability structures social interaction in the prehistoric American southwest}

\author[1,*]{Nicolas Gauthier}

\affil[1]{School of Human Evolution and Social Change, 900 S Caddy Mall, Tempe, USA}

\affil[*]{Nicolas.Gauthier@asu.edu}

\keywords{Archaeology, Spatial interaction, Drought}

\begin{abstract}
  How robust were agrarian social networks to drought? Social networks help absorb weather-related shocks by facilitating resource flows to afflicted settlements and population flows away from them. This property of social networks depends on the degree to which the networks can connect topographically accessible locations that tend to experience different weather patterns. We thus expect rainfall covariance in space and time to interact with patterns of landscape connectivity to structure prehistoric social networks.
\end{abstract}
\begin{document}

\flushbottom
\maketitle


\thispagestyle{empty}


\section*{Introduction}

This is why a solid intro is *so* important for academic articles. And why I recommend a 3-paragraph model:
1. why this study is needed (preview background)
2. what this study does/finds (preview methods/findings)
3. what this study contributes (preview discussion)

In times of drought and famine, farmers in dryland environments turn to their social networks to avoid starvation. Food transfers among farmers link the food supplies of distant settlements. Similarly, atmospheric transport of moisture via local evapotranspiration and re-precipitation can sync crop yields across distant agroecosystems. Tracing the flows of food, water, and energy within these complex social-ecological systems is essential for understanding their long-term behavior, and leveraging our archaeological understanding of why societies succeed or fail will be critical to anticipating the impact of impending climate changes on farming communities in the developing world.

I propose to build models -- statistical, computational, and mathematical -- of food exchange in semiarid environments, and apply these models to an empirical archaeological case study from the pre-Hispanic American Southwest. Rates of site preservation and recovery are exceptionally high in this region. Nearly two centuries of survey and excavation have yielded extensive, high quality settlement pattern data.


   Late pre-Hispanic US Southwest -- Detailed inventories of material culture at nearly 1,000 archaeological sites provide an unparalleled view of the structure and dynamics of past social networks, and the climate of this period has been intensively studied by paleoclimatologists and climate modelers. Here, I will use statistical models to isolate robust social and environmental *patterns* at the macro-scale.


By combining first-principles modeling with extensive empirical datasets, the proposed research is of a kind sorely needed in the ongoing study of sustainability in social-ecological systems. This interdisciplinary modeling framework, and the insights generated with it, will be of use not only to archaeologists, but also to anthropologists, development economists, and the broader climate-change impact-assessment community by making tractable problems with hidden, non-trivial human-environmental linkages.

We explore the dynamics of social networks as scoial infrastructure, a set social relaitonhsips for redistributing food, people, and information that increases the robustness of human populations to environmental variability.


Alternatively, we can think of patterns of interannual variability. After holding mean climate constant, we look at the pattern of excursions from that mean value for each region. So it doesn't matter if you are comparing a wet and dry location, what's important is whether each location is wetter or dryer *than average for that location* at similar times. The benefits of interacting with populations experiencing predictably distinct drought patterns are obios. Social networks allow for the flow of information between spatially distinct populations, allowing agents to monitor conditions in other places, allowing them to target locations for trade or migration.

In general, we expect climate variability to impact population dynamics on a faster time scale than mean climate, as the latter case involves slower scale changes in resource dynamics.

The benefits of social intearctio in this context include information, resources and trade, food excahnge, migration, marriage. Costs include metabolic costs, social costs of moving through unfriendly territory, water, time and related opportunity costs, difficulty of monitoring cooperation and sanctioing freeriders, and imperfect informaiton about where best to go. cite rautmann for idea that actual volumeof food is less imporant that information.

benefits of maintaining social relationships

Food transfers appear to be a cultural universal [@Fe2001,Hill2011Co-ResidenceStructure], yet two distinct strands of research divide anthropological thinking on food-exchange systems. Evolutionary anthropologists, drawing on the behavioral ecology literature, focus on food-sharing behaviors in terms of the evolution of cooperation in small-scale foraging societies, with a general focus on meat sharing on daily time scales [@Gurven2000,Ziker2005, Hames2007, Allen-Arave2008].  Archaeologists drawing from the economics literature focus more on the risk-minimizing aspects of food exchange, with a general focus on the transfer of cereals between agriculturalists on annual to decadal time scales [@Gallant1989,Garnsey1989,Rautman1993a,Hegmon1996].

These are arbitrary disciplinary distinctions. Humans transfer food on a variety of scales and social contexts. Large-scale risk-managing food transfers among agricultural societies are subject to the same kinds of social dilemmas as arise from cooperative behavior in small scale societies, and food-sharing systems minimize subsistence risk regardless of whether risk minimization is why they originally evolved. These two research domains require a unified theoretical approach.

Recent theoretical and empirical work has begun to address how spatial, social, and environmental factors structure food-exchange networks [@Nolin2010Food-SharingIndonesia,Koster2014,Hao2015,Schnegg2015]. Food-exchange systems are not independent of the environment, and the biophysical context of exchange is an important component often ignored in the anthropological literature. A more general approach to these systems views them as a form of social infrastructure, channeling the flow of energy between spatially structured populations in much the same way as food webs channel energy in ecosystems [@Crabtree2015,Crabtree2017ReconstructingStates].

% Food exchange, social networks, and infrastructure

Infrastructure is the filter through which humans interact with their environment [@Anderies2015]. A farmer, for example, does not interact with rainwater directly but instead uses systems of canals and fields in coordination with a network of other farmers to manage flows of water in space and time [@Yu2015]. Canals, roads, and other forms of physical infrastructure enable physical flows like water and people. Social infrastructure channels information, and provides affordances for additional mass and energy flows [@Anderies2015]. At their core, food-exchange networks are clusters of social relationships that redistribute food and people among a structured population of agents. These relationships are a form of public, social infrastructure because they enable the transfer of energy (calories) between populations of potentially distant resource systems (agricultural settlements and their hinterlands). Social networks are both intangible and irregularly mobilized, requiring significant investment to maintain and monitor. As is the case with physical infrastructure, social networks will degrade over time if not actively maintained.

Social infrastructure interacts directly with physical infrastructure because social networks often must map onto spatial networks. Metabolic costs, such as the energy expended producing and transporting food over space, provide constraints on energy flows in exchange systems [@Drennan1984]. In any particular case, the balance between these costs and the metabolic benefits of food transfers influences whether food is moved in bulk to the population in need, or whether that population moves itself to the available food. The topology of spatial networks also constrains who can interact with whom, introducing bottlenecks and other structural flow constraints [@Barthelemy2011SpatialNetworks]. Improvements to transportation infrastructure, such as roads and trails, decrease the effective distance between different settlements; failure to maintain these transportation networks increases the effective distance [@McCall1985TheAfrica].

A canal system cannot be understood in isolation from local weather and topography, and neither can an exchange system. An idealized food-transfer network (i.e., lacking physical or social constraints) acts as a spatiotemporal **low-pass filter** on environmental noise, meaning that farmers in the network receive the space-time average crop yields given variable rainfall [@Harpending1977]. An understanding of the patterns of variability in rainfall, in particular how rainfall covaries across different nodes in the network, will thus provide a insight into the kinds of environmental pressures that drive food-transfer systems [@Rautman1993a,Freeman2014]. The **social-ecological network** concept is useful tool for understanding how social networks fit into a such a broader ecological system.




%The Archaeology of Social Interaction in the American Sotuhwest

Archaeology -- because of its focus on the material correlates of human behavior over long time spans -- is uniquely suited to address how social and physical infrastructure modulates human interactions with the environment. Not only do archaeologists catalogue the remains of field systems, road networks, canals, and other components of hard infrastructure directly, but also the ceramics, raw materials, and luxury goods that are the material correlates of networks of exchange and interaction.

An central principle in archaeology is that we can use the spatial and temporal patterns of environemntal variability to predict ideal cultural responses, and then compare those expected patterns to observed ones to test the importance of that particular pattern of environmental variability. [@halstead1989]. This process gives archaeolgically testable empricial implications -- that people will interact with people who have a different risk profile. The physcal environment defines the costs and benefits of social interaction, and provides the structuring context in which human decision making occurs.

This is a common qualitative observation in many anthropolgical and archaeological stuides of traditional societies (scudder1962, cashdan 2001, waddel 1975, colson 1979, speilmann 1982, 1986, cashdan 1985, wiessner 1982, werner 1983, yengoyan 1972, duff 1998, 2002, cameron 1995, jaskoff and adams 1977, crumly 1979).
There have been fewer qualitiative tests of these patterns in anthropology (with some notable exceptions (Strawhacker, cordell, rautman, johnson)). This study seeks to extend the findings from these specific works to the greater Southwest region, using a much richer colleciton of social and evironemtal data and theory informed by complex systems.

Cordell et all investigated theimpact of preciptaion variability on ceramic exchange in tenth-thirteenth century mesa verde. They highlight a bimodal precipitation regime, extracted from a PC analysis of 28 tree ring cronologies form 689-1988. They show the leading modes of variability follow an east-west split, with a bimodal precipiation regime in the west dominated by pacific ocean, and unimodal rpecipation in the east dominated by cyclonic rainfall form the gulf of mexico. The line of demarcation between regions demoniatd by each mode of variability is a ~100km wide line that varies with the jet stream. The argue that this pattern breaks doesn ca 1239-1488, and explain it as the source of disruption in this period as the social networks that developed to cross it over the centuries were not able to adapt in time to the temproariliy distinct precipitaiton regime. They suggest that broad ceramic regions examined by archaelogists reflect "maximal risk sharing networks."
--how's what im doing differnt -- braoder climatedata, much more robust spatial sampling, much more etensive quantitiative

Rautmann focused on central new mexico, 900 - 1259. Using BR similarity coefficient and local weather station preicpitation data, she found networks up to 50km away. This is consistent with greater inetation within 80km due to spacing required to maintian viable breeding population she says. She highlights differential"investment of social nergy in the maintainance of social ties between the two areas"

Social networks are surivval networks uses the same dataset here, but different because ....
They note that both Zuni and Hopu survive drought

Maize to chaco ... uh just cite i guess


This problem has been the focus of many computaitonal simulations. risk sharing works in pooling network for need based transfers(hao et al 2015) but a key factor was how well eveyrone knoew everyonelses wealth. this is acost od distance, further away you are the les you can monitor others for free riding
janssen pop aggregation shoes climate variaiblity impacts short term aggregation, resource dynamics long term population levels, exchange cricial for maintianin populatiopn levels, sharing leads to short term stability but long term degredation

hegmon - resricted sharing?

crobtree -- importance of asymmetry of debt and obligations -- from this perspective eof's are optimal because you are most garuanteed to minimze the asymmetry of your dbts

anderies and hegmon -- infomraiton contstaints important for migration, need to know that your potential migration desitnation is actually a good place to go, because if you go and you were wrong (its jsut as bad) then thats a big costs. errors in information, such as over distance, make people more likely to move when its actually a bad aidea and fwer to move when its a really good idea

anderies nelseon and freeman - exchagne resources with different trisk profiels -- even in wet environments, an dincrease in interannual variabiliytwill decrease the robustenss of specialist food supply in variaiotoin to rainfall -- so ag economy based on specialization is more limited by interannual deviations than mean amount

--TODO do i need to discuss past least cost path papers in the southwest? probably.

% Spatial interaction models and social gravity
chester king, 1976, 289-290 is source of idea that "the greater the difference between thet wo resource areas, the greater the insenisty of economic interaction and the greater the hnumber of socail bonds, including marriages"

Johnson actually fits a gravity model to chumash marriage data
The use or gravity style statistical models is uncommon in archaeology, with exceptions (johsnon, others he cites). The method is becoming popular in archaeology from a dynamical perspective, but not yet from a statistical perspective.
He finds some evidence for increased marriages between different average climate zones (coastal and inland) but the effect is weak compared to distance and  size

Tobler cappadocian speculation uses it too

To our knowledge this study is the first of its kind to use social gravity models on archy data from statistical perspective

Here, I use two empirical archaeological case studies to explore interactions between food exchange and ecohydrology: the late pre-Hispanic American Southwest and the Roman Imperial period in North Africa. These regions both span areas of about 460,000 square kilometers centered on latitude 35&deg;N, the rough limit of the subtropical ridge that determines the northern extent of the world's hot deserts (Figure \ref{fig:case-studies}). Winter rainfall is delivered by large-scale precipitation and mesoscale storms brought by westerly winds, and summer precipitation falls from convective storms associated with southerly Monsoonal winds. Rainfall in both seasons varies markedly year-to-year and, because the majority of annual precipitation can fall in only a handful of storms, it is highly unpredictable in space. Multidecadal drought conditions are common in these regions. Global atmospheric teleconnections often initiate drought conditions -- unusually cool Pacific sea surface temperatures (La Ni\~{n}a phase of the El Ni\~{n}o-Southern Oscillation) in the American Southwest. Interactions between the land and atmosphere are also strong [@Koster2004RegionsPrecipitation], so the length of these dry spells often reflects more localized positive feedbacks between vegetation and soil moisture [@Ault2014AssessingData]. Vegetation growth in semiarid environments is water limited, but vegetation itself is often a major source of this  water because of terrestrial moisture recycling. Humans in these regions not only depend on these feedback loops -- soil moisture constrains cereal agriculture as much as natural vegetation -- but also play an active role in them through deforestation and irrigation.

Human populations in these environments have developed similar suites of social and physical infrastructure to manage environmental risk. Food storage is one effective strategy for preserving bulk grains in dry environments, and storage features are a common archaeological find [@Spielmann2011SustainableEnvironments]. Irrigation, in particular via runoff harvesting infrastructure near seasonally flooded streams (**wadis**/**arroyos**), redistributes soil moisture in space and time to create microenvironments for agriculture. But these systems are vulnerable to flooding and would have demanded significant labor investments to monitor and maintain [@Shaw1982, Dominguez2005,Beckers2013AncientAsia]. Such strategies would have been effective for managing small-scale variability (year-to-year and field-to-field), but would have been vulnerable to a long-lasting, spatially extensive droughts [@Halstead1989]. During such extreme weather events, inhabitants of these regions must mobilize social networks to -- depending on the scale and severity of the event -- move food to afflicted settlements or move people away from them. The precise nature of these social networks, and the means by which this social infrastructure was provisioned, varies substantially between the case studies.

The populations of the late pre-Hispanic period in the American Southwest subsisted mainly on maize production, supplemented by a mix of beans and wild proteins. Food transfers are thought to have occurred primarily in the context of informal sharing within kin groups, reciprocal exchange at ritual ceremonies and festivals, and residential mobility on the scale of one to three generations [@Hegmon1991,Hegmon1996,Kohler1996TheAnasazi,Varien1999SedentismBeyond,Cordell2007MesaMigration]. The archaeological record attests to extensive exchange networks of durable goods such as ceramics and obsidian [@Mills2013a], and there is direct (if limited) evidence for the long-distance transport of maize [@Benson2009PossibleMexico,Benson2010WhoDrought].


Shared
regional styles of elaborate ceramic tableware suggest a major role
for commensality in cultivating relationships of solidarity, both
within and between settlements (e.g., Tomkins, 2007; Urem-Kotsou
and Kotsakis, 2007). -- talk about this in the context of ceramic similarity -- tablewares important because they speek to ocmmensalism and feastin

\section*{Results}

\subsection*{Modes of Drought and Pluvial Variability}

The goal here is to extract ROBUST patterns of climate variability, moving beyond the small-scale point based year to year correlations. The latter can easily be dominated by noise or other issues with sampling variability. The method we use here yields similar results in appearance, but in a more objective fashion designed specifically to skillfully extact signalfrom noise.
Note that we refer to the full range of moisture/ aridity variability here as drought for concreteness and simplicity.The central idea here is that drought and pluvials

![Variance explained by the leading modes of variability](../figures/variance_explained.pdf)

The visible drought patterns are largely consistent with those from other studies using varied observational data and time windows. The leading three eofs, which together explain `r #variance` percent of the variance represent the influence of Pacific sea surface temperatures, and associated anomalies such as El Nino Southern Oscillation and the Pacific Decadal Oscillation. The fourth and fifth empriicial orthogonal functions represent internal atmospheric variability, associated with regional circulation patterns not influenced by the pacific. spatially coherent

![](../figures/reof_observed.pdf)

basic hypothesis comes down to climate patterns in terms of averages or interannual variability. in practice, it comes down to an R-mode vs an S-mode pca

spei is indiependend of local climatology

leading reof looks like el nino, resembles southern oscialltion pattern (antiphase) from mccabe and dettinger 1999

distinguish between forced spatial patterns from ssts and those that are spatilly coherent internal variability, "noise" in some sense, but distinct from the spatially incoherent variability

reofs not orthogonal in spcae, but they are in time, what does that mean for thinking about interaction?

leading 3 eigenvalues expalin nearly 70% of the variability and reflect ocean, consistent with mccabe et al 2004
the leading 2 specifically show up  in cook and meko
resembles varimax rotate pcs from cook meko et al 1999 j climate -- summer pdsi variability

Temperature and rpecipitaiton averages are almost entirely determined by precipitation, so we control? for that, or at least distcount it

REOF1 southwest sonoran desert zone

\subsection*{Spatial Interaction Modeling}

here talk about best fitting models and whatnot.

\subsubsection*{Distance Decay of Social Interaction}

Look at distance cutoffs, are they consistent with the figures of 50km-80km from rautman and drennen and others? johnson finds no marriages past 60km, maize to chaco papers also have distance figure of 80km

"plateu effect" or distance within which distance doesn't matter, from johnson but cites olsson 1965 and crumley 1979. also cite ariadne distance deterrence function for similar idea
~5km should be the ticket if marchetti's constant holds

\subsubsection*{Drought Variability and Social Interaction}

interpret relative r2 -- in johnsons chumash paper r2 was .43 for distance and size, and adding dummy variable for same or different enviroment only increases model to .44

do negative interactions (ie low) point to raiding and conflict?

zuni and hopi

changes in rho value over time -- pointst tomore idiosyncratic interaction

Up to three levels of \textbf{subheading} are permitted. Subheadings should not be numbered.

\section*{Discussion}


points to disucss:
1. Why doesn't size matter?
2. Why do things change over time?
3. What does ceramic assemblage similarity even mean?
3b. fully connected widhgted network, but what if pruned?
  we argue here our approach here is apporaprate because we are aggregating many, sparse social networks of indivudals together at the population scale. that is we look at macroscale flow patterns given microscale social networks, maxent approach means we make the fest possible assumptiopns about the configuration of the microscale social networks
5. Importance of asymmetry. citing crabtree talk about asymmetric debts. also talk about differences in population size and gravity model. what do we miss out on when use symmetric data?
4. zuni and hopi? seemingly different because different external vs internal tie patterns, but actually they both cross climate zones
6. we just looked at summer growing season, but what about winter (where bimodal)? Site coates et al 2015 for variability in the seasonal phasing of rainfall
7. We don't look at the role of drought in producing simmilarity, such as a drought leading to many migrants, but rather look at long term patterns in the differences. This is because we integrate out the mean of each site to get attractivities, effectively eintegrating out utility measures (bavaud 2002, 2008). So we are specifically looking at the "deterrence" effect of claimate distances, that is the costs and beneftis of interaction.

## Qualifications, Caveats, and Future Work
jensen inequality, E(logit(y)) >=logit(E(y)). this means we should interpret specific predictions and confidence intervals from this model with caution, and instead only focus on interpretaiton of the brod functional forms. Its a necessary evil here, becuase we do not yet have the computational abiliy to fit the corMLPE correlation structure and a beta family. Preliminary tests with the data suggested that the pairwise correlation structure was a much alrger source of bias then transforming the response variabble,so that's the tradeoffwe went with. The bias from transforming the respone leads to bias in the estimates of variance

the eof patterns are in part scale dependent. Thihs iss isnt necesarily a problem, its a necessity that the answer to the question what kind of variability matters is inherently tied to the scale at which on asks the question, there are no right or wrong answers. Although the particular choices of variables, resolution, domain size, truncation level, were informed here by theory and tested to as to minimze sensitivity to paritcular choices made by the authors, different researchers could still generate equally resonable results given  different input parameters.  This is simply to state that the drought patterns highlighted here, while we are confident they are not statistical artifacts, should be used with caution in contexts outside of the specific context here.

\section*{Methods}


\subsection*{Spatial interaction model}

We start with a symmetric similarity matrix $\mathbf{T}$ with elements $T_{ij} = T_{ji} = BR_{ij}$ with $BR$ defined as the scaled Brainerd-Robinson coefficient of similarity between the proportions of decorated ceramic ware types.

$$BR = 1 - \sum_{k=1}^{p} \lvert P_{ik} - P_{jk} \rvert$$

where P_{ik} is the proportion of ceramics of type $k$ at site $i$.

This metric estimates the similarity in the ceramic discard assemplages between pairs of sites, which in turn is a proxy for cultural preferences, food consumption differences, access to raw materials, and access to trade networks. The index can be loosely interpreted as a probability of interaction between two sites, with identical patterns of ceramic discard indicating a high likelihood of interaction via either direct migration or trade or indirect cultural diffusion.

This is a standard metric in archaeology (cowgill 1990, golitko et al 2009, hart engelbrecht 2012), similar to the xy distances used in other fields, and is used to .

We fit models of the form

$$T_{ij} = k v_i^\mu w_j^\alpha \exp(\beta c_{ij})$$

where $c_{ij}$ is some notion of the generalized cost of moving from $i$ to $j$ and is symmetric, meaining $c_{ij} = c_{ji}$.

Gravity models of this form are quasisymmetric, and can be decomposed into symmetric and asymmetric components


Which states that the flow of people, goods, or information from site $i$ to site $j$ is a function of some property $v$ of site $i$, property $w$ of site $j$, and the distance between them $c_{ij}$ with $c_{ij} = c_{ji}$

A key question is how to relate directed, asymmetric flows in $T_{ij}$ to the undirected, symmetric similarities in $BR$, that is what is $BR = f(T_{ij},T_{ji})$, which is to say how does ceramic assemblage similarity reflect the underlying flow of people, goods, and information?


Question: Do regional interaction networks self-organize with respect to patterns of climate variability, and if so at what spatial scales does this organization occur?

Social networks help absorb weather-related shocks by facilitating resource flows to afflicted settlements and population flows away from them. This property of social networks depends on the degree to which those networks can connect topographically accessible locations that tend to experience different weather patterns. I thus expect patterns of rainfall covariance in space and time to interact with patterns of landscape connectivity to structure prehistoric social networks.

This relationship has been documented in the US Southwest on smaller spatial scales, using definitions of ``climate variability'' limited by the environmental data available in each case [@Rautman1993a,Cordell2007MesaMigration,Strawhacker2017RiskProvince]. The proposed work expands on these efforts to analyze the entire regional system, using two types of climate variability derived from a paleoclimate model-data fusion technique.

\subsection*{Archaeological Social Network Proxies}

I draw on an extensive archaeological database from the US Southwest (Figure \ref{fig:swsn}) to address this question. The **Southwest Social Networks** (SWSN) database is a compendium of material-culture data from nearly 1,000 well-dated sites in Arizona and western New Mexico from between 1200 and 1500 CE [@Mills2012,Mills2013a,Peeples2013,Borck2015,Hill2015,Mills2015a]. Drawing on a large sample of sites from the earlier **Coalescent Communities** database of all recorded prehistoric settlements with more than 12 rooms in the Southwest from 1200 to 1700 CE [@Hill2004], the SWSN project analyzed nearly 4.7 million ceramic artifacts and nearly 5,000 obsidian artifacts [@Mills2015a]. Using an index of the similarity of ceramic assemblages and geochemical sourcing as proxies for the intensity of social interaction between settlements, the SWSN database provides quantitative estimates of the topology of the region-wide social network during six 50 year time steps [@Mills2013a].

\subsection*{Topography and Least-Cost Networks}
All else being equal, locations that are closer together in space will be more similar than those further apart [@Tobler1970]. Modelling this spatial structure in the SWSN data is necessary to control for spatial dependence in statistical analyses of these data. To accomplish this, I calculate the least-cost network between all sites in the SWSN network. This method provides an efficient estimation of the metabolic costs of movement between any pair of settlements in the network. I calculate these costs using the Pandolf-Santee formula [@White2012], which relates the metabolic rate of a traveler in watts to the traveler's weight and external load, walking speed, terrain slope, and a dimensionless terrain roughness coefficient. Using energy expenditure rather than time or Euclidean distance to represent movement costs facilitates a direct comparison of the metabolic costs and benefits of food transfers  [@Drennan1984]. The resulting least-cost network will be used as a proxy measure for the constraints on moving both people and bulk goods across the landscape, and thus the topographic affordances for social exchanges.

\subsection*{Paleoclimate Data Assimilation}
ccsm simulates monsoon ok, better than others
Estimates of past hydroclimate in the American Southwest are generated using paleoclimate modeling tools, in particular Earth system model simulations, data assimilation, and bias correction and spatial downscaling. Climate-model simulations are a valuable method to estimate past, present, and future climate states. Climate models generate physically consistent climate fields at high temporal resolutions. These models capture the dynamic response of the Earth’s climate system to external forcing and internal variability by resolving a series of differential equations governing atmospheric and oceanic flow on a three-dimensional mesh of points [@Gettelman]. Recent state-of-the-art **Earth system models** extend this approach by simulating the interactions between the atmosphere, ocean, ice, land, and biosphere, including an array of biogeohphysical and biogeochemical processes such as the precipitation-vegetation feedbacks generating long-term droughts [@Hurrell2013,Kay2015,Gettelman].

Data assimilation techniques such as the ensemble Kalman filter are used to generate atmospheric \textit{reanalyses}. A reanalysis is a model-based climate hindcast optimally constrained by instrumental observations. Paleoclimate data assimilation relies on the same principle but replaces instrumental observations with paleoclimate proxy records such as tree rings, ice cores, and corals [@Hakim2016TheResults]. Paleoclimate model simulations serve as physically consistent prior information about the possible states of the climate system, and the proxy records provide novel information about the time evolution of those states. The ensemble Kalman filter uses this information, as well as the spatial covariances from the climate-model prior (e.g. teleconnections), to "spread out" information from the point-based proxies to generate physically optimal extrapolations beyond the proxy locations [@Acevedo2015TowardsTechniques,Hakim2016TheResults].

For the late pre-Hispanic period of interest, I will use outputs from the Community Earth System Model (CESM) Last Millennium Ensemble (LME) as a model prior. LME is a paleoclimate simulation that uses CESM to recreate the transient response of the global climate system to changes in climate forcings (e.g. orbit, greenhouse gasses, volcanic activity, land-cover change) during the past 1,000 years [@Otto-bliesner2015]. The model was run repeatedly from different initial conditions to assess the range of internal model variability to fixed forcing mechanisms.

\subsection*{Statistical inference}

To isolate meaningful patterns of variability in the paleo-reanalysis, I use the model outputs to calculate a drought index and then decompose the nearly 3,600 monthly drought maps from 1200 to 1500 CE into five \textit{empirical orthogonal functions} (EOFs) (Figure \ref{fig:eofs}). EOFs are the eigenvectors of the climate space-time covariance matrix, such that the leading EOFs capture most of the variability in the original climate signal [@Lorenz1956EmpiricalPrediction]. EOFs are equivalent to the principal component analysis commonly used in archaeology \cite[e.g.][]{Dean1996DemographyStress], save for that the principal components of a spatiotemporal dataset only capture temporal signals.

varimax rotation, scaled by the square root of the corresponding eigenvalues

I will then use nonlinear regression (generalized additive models for beta-distributed data [@Wood2006a]) to determine whether the patterns of interaction strength from the SWSN database are significantly related to the patterns of variability captured in the EOFs (Figure \ref{fig:eofs}). I will compare the following hypotheses:

Null Hypothesis -- Social networks are not related to climate patterns, and the strength of interaction between any two sites is solely a function of their relative sizes and the distance between them.
Hypothesis 1 -- Interactions between sites in opposing EOFs are significantly stronger than would be expected by distance alone.
Hypothesis 2 -- Interactions between sites experiencing different mean growing-season climates are significantly stronger than would be expected by distance alone.

We fit generalized additive mixed models, and select the most parsimonious model from all candidate models using Akaike's Information Criterion. AIC selection between linear mixed models outperformed several other estimating techniques in landscape genetics in simulation studies by [@Shirk et al 2018]. We account for nonindependence of ovserved edges that share origin or destination sites using the maximum likelihood population effects correlation structure, which models the co dependence of errors on shared sites. This helps account for the nature of pairwise data, including overpowered.

Archaeological networks have several specfic features that can make statistical inference difficult.



\bibliography{sample}

\noindent LaTeX formats citations and references automatically using the bibliography records in your .bib file, which you can edit via the project menu. Use the cite command for an inline citation, e.g.  \cite{Hao:gidmaps:2014}.

For data citations of datasets uploaded to e.g. \emph{figshare}, please use the \verb|howpublished| option in the bib entry to specify the platform and the link, as in the \verb|Hao:gidmaps:2014| example in the sample bibliography file.

\section*{Acknowledgements (not compulsory)}

Acknowledgements should be brief, and should not include thanks to anonymous referees and editors, or effusive comments. Grant or contribution numbers may be acknowledged.

\section*{Author contributions statement}

Must include all authors, identified by initials, for example:
A.A. conceived the experiment(s),  A.A. and B.A. conducted the experiment(s), C.A. and D.A. analysed the results.  All authors reviewed the manuscript. 

\section*{Additional information}

To include, in this order: \textbf{Accession codes} (where applicable); \textbf{Competing interests} (mandatory statement). 

The corresponding author is responsible for submitting a \href{http://www.nature.com/srep/policies/index.html#competing}{competing interests statement} on behalf of all authors of the paper. This statement must be included in the submitted article file.

\begin{figure}[ht]
\centering
\includegraphics[width=\linewidth]{stream}
\caption{Legend (350 words max). Example legend text.}
\label{fig:stream}
\end{figure}

\begin{table}[ht]
\centering
\begin{tabular}{|l|l|l|}
\hline
Condition & n & p \\
\hline
A & 5 & 0.1 \\
\hline
B & 10 & 0.01 \\
\hline
\end{tabular}
\caption{\label{tab:example}Legend (350 words max). Example legend text.}
\end{table}

Figures and tables can be referenced in LaTeX using the ref command, e.g. Figure \ref{fig:stream} and Table \ref{tab:example}.

\end{document}